\documentclass[margin,line]{res}
\usepackage{hyperref}
\usepackage{url}
\oddsidemargin -.5in
\evensidemargin -.5in
\textwidth=6.0in
\itemsep=0in
\parsep=0in
\topmargin=0in
\topskip=0in
 
\newenvironment{list1}{
  \begin{list}{\ding{113}}{%
      \setlength{\itemsep}{0in}
      \setlength{\parsep}{0in} \setlength{\parskip}{0in}
      \setlength{\topsep}{0in} \setlength{\partopsep}{0in}
      \setlength{\leftmargin}{0.17in}}}{\end{list}}
\newenvironment{list2}{
  \begin{list}{$\bullet$}{%
      \setlength{\itemsep}{0in}
      \setlength{\parsep}{0in} \setlength{\parskip}{0in}
      \setlength{\topsep}{0in} \setlength{\partopsep}{0in}
      \setlength{\leftmargin}{0.2in}}}{\end{list}}


    
\begin{document}

\name{\LARGE John L. Waczak} \hfill {\em \today}

\begin{resume}
\section{\sc Overview}

\vspace{.05in}
\begin{tabular}{@{}p{3.5in}p{3in}}
Undergraduate Researcher              & {Phone:}  (503) 330-1280 \\
Roundy Research Group - Oregon State University 
 & {E-mail:}  waczakj@oregonstate.edu\\
12231 SW Pond Lane\\
King City, OR, 97224  & {LinkedIn:} \scriptsize{ \url{www.linkedin.com/in/john-waczak/}}
\end{tabular}


\section{\sc Interests}

Computational Physics, Applied Mathematics, Spectroscopy

\section{\sc Education}
{\bf Oregon State University}, Corvallis, OR. \hfill September 2015 -- present\\
%\vspace*{-.1in}
Bachelor of Science in Physics and Mathematics \hfill(GPA 3.90/4.0)

\section{\sc Technical Skills}
{\bf Programming and Scripting}: Python, C++, C\#, Mathematica, Microsoft Word, Microsoft Excel, Numpy, Matplotlib, SQLite, PID control, Modbus serial protocol \\
{\bf Electronics}: Arduino, Raspberry Pi, Nation Instruments, Soldering, basic circuit design 
%%%%%%%%%%%%%%%%%%%
\section{\sc Experience}
%%%%%%
{\bf Roundy Research Group}, Oregon State University \hfill{January 2017 -- Present}\\
{\em Undergraduate Researcher}\hfill \scriptsize{\url{github.com/john-waczak/dynein_walk}}\\
I am working with Dr. Roundy to develop and test computational simulations which evaluate a physical model for the locomotion of the cellular motor protein Dynein. My duties include editing simulation code (C++), writing scripts to analyze and visualize large data sets (python), exploring the simulation space to find the best set of parameters to produce \textit{physical} behavior, and presenting work at weekly group meetings.
%%%%%%

{\bf Cooper Environmental Services}, Beaverton, OR \hfill{March 2015 -- September 2017}\\
{\em Intern} \\
I Worked in the lab to evaluate the physical and chemical properties of various filter media in order help the company decide which filter to use in the Xact Ambient Metals Monitor. I designed and implemented controller logic (using a PID loop) for controlling a vacuum pump through c\# in conjunction with a National Instruments IO board. Using Modbus serial communications protocol, I created code to make data accessible locally with a Modbus slave device. I wrote code using SQLite to send the same data to a local database for long term storage. I worked with production to solder wire assemblies that are used in the Xact. 
%%%%%%%%%%%%%%%%

{\bf Physis 199 TA}, Oregon State University \hfill{January 2017 - April 2017} \\
I served as one of two undergraduate TA's for the OSU PH199 class. I helped coordinate lab tours, talks from professors, and resume / career workshops. In the final class, I helped organize a panel of my fellow upper-division students to allow the class to ask questions about the major and what to expect. 

%%%%%%%%%%%%%%%%

{\bf Physics Exam Proctor}, Oregon State University \hfill{September 2016 -- Present}\\
During the midterms and finals weeks I serve as an exam proctor. I meet with the professor, go over exam questions as well as suitable answers to forecasted student questions, and then go to the exam room. There I am in charge of passing out tests and ensuring there is no cheating. 

\section{\sc Activities}
\begin{itemize} 
	\item {\bf Lambda Chi Alpha ($\Lambda XA)$} Vice President \hfill September 2015 to Present\\
	I am the current Vice President of the Alpha Lambda chapter of Lambda Chi Alpha Fraternity. We are a 140 man organization devoted to building leaders and serving our community. My duties as VP include organizing weekly brotherhood event, presenting each week during our chapter meeting, talking with Alumni, and sitting on our executive committee. 
	
	\item {\bf Society of Physics Students} Member, Mentor \hfill September 2015 to present\\ 
	As a mentor, I am paired with a new undergraduate physics major whom I help acclimate to university classes (expectations, resources, class suggestions, etc...). We also actively engage the community through outreach events like the August 2017 eclipse viewing.
	
	\item {\bf Sigma Pi Sigma ($\Sigma \Pi \Sigma$)} National Physcis Honor Society member \hfill June 2017 to present
	
	\item {\bf Community Outreach Inc.} Volunteer \hfill September 2015 to present\\
	Helped raise over 300,000 lbs. of food and \$30,000 for local families through through my fraternity's philanthropic events: “Watermelon Bash”, "Caddyshacked", "Can-you-dash".
	
\end{itemize} 

\section{\sc Awards}
\begin{itemize}
	\item {\bf Lambda Chi Alpha - Alpha Lambda Chapter Freshmen of the Year} \hfill June 2016\\
	Present for grades, involvement, and overall chapter vote.
	
	\item {\bf Oregon State honor roll} \hfill 2015-present\\
	Presented for achieving above a 3.5 GPA and being enrolled in a minimum of 12 credit hours per term. 
\end{itemize}
\section{\sc References }
Available upon request.

\end{resume}
\end{document}




